\documentclass[11pt]{article}
\usepackage[utf8]{inputenc}
\usepackage[IL2]{fontenc}
\usepackage[czech]{babel}
\usepackage{listings}
\usepackage[spaces,hyphens]{url}
\usepackage{amsmath,amsfonts,amssymb}
\usepackage{graphicx}
\usepackage{tikz}

\setlength{\parskip}{1em}
\setlength{\parindent}{0em}
\setlength{\oddsidemargin}{24pt}
\setlength{\evensidemargin}{24pt}
\setlength{\textwidth}{421pt}
\tikzstyle{every node}=[circle, draw, fill=black!50, inner sep=2pt, minimum width=8pt]

\begin{document}
    \part*{Prove that for given $n, m \in \mathbb{N^+}, n | m$ there is an isomorphism between $n\mathbb{Z}/m\mathbb{Z}$ and $\mathbb{Z}_{m/n}$ }

    $n\mathbb{Z}/m\mathbb{Z} = \{a + m\mathbb{Z} | a \in n\mathbb{Z}\} = \{\{a \equiv b \mod m | b \in n\mathbb{Z}\} | a \in n\mathbb{Z}\} = \{\{na \equiv b \mod m | b \in n\mathbb{Z}\} | a \in \mathbb{Z}\} = \{\{na \equiv b \mod m | b \in n\mathbb{Z}\} | a \in \mathbb{Z}_{m/n}\}$
    (Because $n \times (a + m/n) \equiv na \mod m$)
    Which we can use to define the isomorphism.

    $h: \mathbb{Z}_{m/n} \leftrightarrow n\mathbb{Z}/m\mathbb{Z}; h(a) = \{na \equiv b \mod m | b \in n\mathbb{Z}\}$
    This gives different answers for different $a$'s and it covers the set
    $n\mathbb{Z}/m\mathbb{Z}$.

    For $a + b = c$ in $\mathbb{Z}_{m/n} \longrightarrow \{na \equiv x \mod m | x \in n\mathbb{Z}\} + \{nb \equiv x \mod m | x \in n\mathbb{Z}\} = \{nc \equiv x \mod m | x \in n\mathbb{Z}\}$ is expected behavior and applies other way around as well.

    So it is a well defined isomorphism. $\square$
\end{document}
