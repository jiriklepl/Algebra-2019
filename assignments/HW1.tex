\documentclass[11pt]{article}
\usepackage[utf8]{inputenc}
\usepackage[IL2]{fontenc}
\usepackage[czech]{babel}
\setlength{\parskip}{1.5em}
\setlength{\parindent}{0em}
\usepackage{a4wide}
\usepackage{listings}
\usepackage[spaces,hyphens]{url}
\usepackage{amsmath,amsfonts,amssymb}
\usepackage{graphicx}

\begin{document}
    \part*{Proof that a semigroup is a group if it has a unit element and it is left-cancelative and left-divisible}
    
    Let $S(\cdot)$ be the semigroup and $e \in S$ its unit element. \\
    We define table $T; (\forall row, col \in S)(T(row, col) := row \cdot col)$.

    We know that $S$ is left-cancelative iff $(\forall r, a \in S)(|\{(r, c); c \in S | T(r, c) = a\}| \le 1)$. \\
    And that $S$ is left-divisible iff $(\forall r, a \in S)(|\{(r, c); c \in S | T(r, c) = a\}| \ge 1)$.

    Which means that: $(\forall r, a \in S)(|\{(r, c); c \in S | T(r, c) = a\}| = 1)$. \\
    And that means: $(\forall r \in S)(T(r, \_) \text{ is a permutation of } S)$. \\
    And that means: $(\forall c \in S)(\exists !c^{-1} \in S)(e = c^{-1} \cdot c)$ and $(\forall c^{-1} \in S)(\exists !c \in S)(e = c \cdot c^{-1})$ \\
    and thus $c^{-1}$ is a well defined inverse of $c$.

    $S(\cdot)$ is a semigroup and it fulfills the existence of unit element and the existence of inverses and therefore it is a group. $\square$

    \section*{Proof that ... and right-divisible}

    Right-divisible: $(\forall a, r \in S)(\exists c \in S)(a = r \cdot c)$ \\
    Left-cancelative: $(\forall a, b, c \in S)(a \cdot b = a \cdot c \Rightarrow b = c)$

    So if there are two distinct values $c_1$, $c_2$ given some $a$ and $r$ for which the right-divisibility formula holds, then they have to be identical (from left-cancelativity). Therefore there cannot be said distinct values.

    With that said we can conclude that $(\forall r, a \in S)(|\{(r, c); c \in S | T(r, c) = a\}| \ge 1)$ and thus that $S(\cdot)$ is also Left-divisible and so we can use the first proof. $\square$

    \pagebreak

    \part*{Proof that a semigroup has a unit element and it is left/right-cancelative and left/right-divisible if it is a group}

    Let $S(\cdot)$ be the given semigroup (group).

    Existence of a unit element is trivial as groups are defined by having a unit element, let's call the unit element $e$.

    From associativity and existence of inverses: \\
    $(\forall a, c \in S)(a = a \cdot e = a \cdot (c^{-1} \cdot c) = (a \cdot c^{-1}) \cdot c) \Rightarrow S$ is left-divisible. \\
    Similarly: $(\forall a, r \in S)(a = \cdots = r \cdot (r^{-1} \cdot a)) \Rightarrow S$ is right-divisible.

    From associativity and existence of inverses and given some $a, b, c \in S$ such that $(a \cdot b = a \cdot c)$:
    $a \cdot b = a \cdot c \Rightarrow a^{-1} \cdot a \cdot b = a^{-1} \cdot a \cdot c \Rightarrow (a^{-1} \cdot a) \cdot b = (a^{-1} \cdot a) \cdot c \Rightarrow e \cdot b = e \cdot c \Rightarrow b = c$ \\
    Thus $S$ is left-cancelative.

    Similarly we can show that $S$ is right-cancelative by reversing the order of terms in each side of every equation.
\end{document}
