\documentclass[11pt]{article}
\usepackage[utf8]{inputenc}
\usepackage[IL2]{fontenc}
\usepackage[czech]{babel}
\usepackage{listings}
\usepackage[spaces,hyphens]{url}
\usepackage{amsmath,amsfonts,amssymb}
\usepackage{graphicx}
\usepackage{tikz}

\setlength{\parskip}{1em}
\setlength{\parindent}{0em}
\setlength{\oddsidemargin}{24pt}
\setlength{\evensidemargin}{24pt}
\setlength{\textwidth}{421pt}
\tikzstyle{every node}=[circle, draw, fill=black!50, inner sep=2pt, minimum width=8pt]

\begin{document}
    \part*{For a prime $p$ prove that $\Phi(p^m) = p^m - p^{m-1}$}

    Definition: $\Phi(n) = \sum\limits_{i=1}^n I(gcd(i,n) = 1)$

    We can modify it as follows: $\Phi(p^m) = \sum\limits_{i=1}^{p^m} I(gcd(i,p^m) = 1) = p^m - \sum\limits_{i=1}^{p^m} I(gcd(i,p^m) > 1) = p^m - \sum\limits_{i=1}^{p^m} I(gcd(i,p^m) > 1) = p^m - |\{p, 2p, \dots p^m\}| = p^m - |\{1, 2, \dots p^{m-1}\}| = p^m - p^{m-1} = p^{m-1}(p-1) \square$

    \part*{For two natural numbers $a$ and $b$ such that $gcd(a, b) = 1$ prove that $\Phi(ab) = \Phi(a) \Phi(b)$}

    WLOG we can suppose that $b$ is of a form $p^m$ where $p$ is a prime, if it holds for this additional requirement then we can decompose arbitrary $b$ to a product over set of these forms and induce over such set.

    If $a$ can be decomposed we can use the same logic (we just switch names of $a$ and $b$) and reduce the problem to:

    $\Phi(p^m q^n) = \Phi(p^m) \Phi(q^n) = p^{m-1} q^{n-1} (p-1)(q-1)$ where $p^m$ is $a$ and $q^n$ is $b$.

    $\Phi(p^m q^n) = \sum\limits_{i=1}^{p^m q^n} I(gcd(i,p^m) = 1 \wedge gcd(i,q^n) = 1) = \sum\limits_{i=1}^{p^m q^n} I(gcd(i,p) = 1 \wedge gcd(i,q) = 1) = p^m q^n - \sum\limits_{i=1}^{p^m q^n} I(gcd(i,p) > 1 \vee gcd(i,q) > 1) = p^m q^n - |\{p,2p,\dots p^m q^n\}| - |\{q,2q,\dots p^m q^n\}| + |\{pq,2pq,\dots p^m q^n\}| = p^m q^n - p^{m-1} q^n - p^m q^{n-1} + p^{m-1} q^{n-1} = p^{m-1} q^{n-1} (pq - p - q + 1) = p^{m-1} q^{n-1} (p-1) (q-1) \square$
\end{document}
