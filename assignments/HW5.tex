\documentclass[11pt]{article}
\usepackage[utf8]{inputenc}
\usepackage[IL2]{fontenc}
\usepackage[czech]{babel}
\usepackage{listings}
\usepackage[spaces,hyphens]{url}
\usepackage{amsmath,amsfonts,amssymb}
\usepackage{graphicx}
\usepackage{tikz}

\setlength{\parskip}{1em}
\setlength{\parindent}{0em}
\setlength{\oddsidemargin}{24pt}
\setlength{\evensidemargin}{24pt}
\setlength{\textwidth}{421pt}
\tikzstyle{every node}=[circle, draw, fill=black!50, inner sep=2pt, minimum width=8pt]

\begin{document}
    \part*{Let $\alpha: R \rightarrow S_8, \beta: R \rightarrow S_{12}, \gamma: R \rightarrow S_6$, be maps given by assigning to
    rotations of a cube permutations of vertices, edges, faces, respectively.
    Prove that $\alpha(R) \subseteq A_8, \beta(R) \subseteq A_{12}, \gamma(R) \subseteq A_6.$}

    We can define a rotation on odd number of elements using even number of transpositions: \\
    Left: $\forall n \in \mathbb{N}, 2 \not| \ n: L(a_1, \dots a_n) = \prod\limits_{i = 1, i := i + 1}^{n - 1} (a_{i}, a_{i + 1})$ \\
    Right: $\forall n \in \mathbb{N}, 2 \not| \ n:R(a_1, \dots a_n) = \prod\limits_{i = n - 1, i := i - 1}^{1} (a_{i + 1}, a_{i})$

    We will use this for rotations of edges and vertices of the cube as any rotation of the cube can be expressed as a product of ``rolls'' ($90^\circ$ rotations around opposite faces).

    One such ``roll'' rotation (no matter whether of edges or vertices) can be expressed as a permutation $(1, 2, 3, 4) \cdot (5, 6, 7, 8)$ where the numbers represent edges or vertices around each of faces we are rolling the cube around.

    $(1, 2, 3, 4) \cdot (5, 6, 7, 8) = R(1, 2, 4) \cdot R(6, 7, 8) \cdot (3, 4) \cdot (5, 6)$.

    After the first two $R$ operations, first two and last two indexes have the right elements and then we need just swap the third with fourth and fifth with sixth.

    Therefore $\alpha(R) \subseteq A_8, \beta(R) \subseteq A_{12}$

    For faces we want $(1, 2, 3, 4)$. $(1, 2, 3, 4) = R(2, 3, 4) \cdot (1, 2)$ which is an odd permutation so $\gamma(R) \not\subseteq A_6$.

\end{document}
