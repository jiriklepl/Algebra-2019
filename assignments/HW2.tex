\documentclass[11pt]{article}
\usepackage[utf8]{inputenc}
\usepackage[IL2]{fontenc}
\usepackage[czech]{babel}
\usepackage{listings}
\usepackage[spaces,hyphens]{url}
\usepackage{amsmath,amsfonts,amssymb}
\usepackage{graphicx}

\setlength{\parskip}{1em}
\setlength{\parindent}{0em}
\setlength{\oddsidemargin}{24pt}
\setlength{\evensidemargin}{24pt}
\setlength{\textwidth}{421pt}

\begin{document}
    \part*{Consider the first definition of a group. Prove that sub-universes of of a group are exactly sub-groups.}
    
    $G(\cdot)$ is a group when:
    \begin{itemize}
        \item $G$ is a set and $\cdot$ is a binary operation on $G$.
        \item $\cdot$ is associative on $G$.
        \item $G(\cdot)$ is left and right divisible.
        \item $G(\cdot)$ is left and right cancelative.
    \end{itemize}

    \section*{Proof:}

    Let $G(\cdot)$ be a group and $S(\cdot)$ its sub-universe. The first two constraints of the definition trivially hold for $S(\cdot)$ from the definition of $G(\cdot)$ and $S$ being its sub-universe.

    The last constraint also holds trivially.

    If we define $T(G, \cdot)$ the table of an operation $\cdot$ on a set $G$ as $(\forall a, b \in G) T(G, \cdot)_{a, b} := a \cdot b$ then $G$ is left and right divisible iff every $a \in G$ is mentioned in each row and column at least once. It is left and right cancelative iff ... at most once.

    We know that $S$ is a subset of a finite set and thus it is also finite. Then $T(S, \cdot)$ is a table with $|S|$ elements in each of its row and column and every $a \in S$ is mentioned at most once in each row and column (as we know $S$ is left and right cancelative) and thus we can conclude that it is mentioned exactly once and therefore divisibilities also hold. $\square$
\end{document}
