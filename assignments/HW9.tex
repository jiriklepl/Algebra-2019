\documentclass[11pt]{article}
\usepackage[utf8]{inputenc}
\usepackage[IL2]{fontenc}
\usepackage[czech]{babel}
\usepackage{listings}
\usepackage[spaces,hyphens]{url}
\usepackage{amsmath,amsfonts,amssymb}
\usepackage{graphicx}
\usepackage{tikz}

\setlength{\parskip}{1em}
\setlength{\parindent}{0em}
\setlength{\oddsidemargin}{24pt}
\setlength{\evensidemargin}{24pt}
\setlength{\textwidth}{421pt}
\tikzstyle{every node}=[circle, draw, fill=black!50, inner sep=2pt, minimum width=8pt]

\begin{document}
    \part*{We have $n$ kinds of beads, each different color and unlimited number of such beads. How many different necklaces of 12 beads can we make?}

    We have an alphabet $A = {1, \dots n}$ and any word consisting of letters from $A$ defines a necklace, set of such words is $W$ (also contains empty word $e$).

    $W(.)$ follows associativity and has an identity element $e$ so it is a monoid with generator $A$.

    Let's define length $l: W \rightarrow \mathbb{N}$ such that $l(e) = 0, \forall w \in W, a \in A: l(wa) = l(w) + 1$.

    Let $G$ be cyclic necklace group on with rotations with a generator $g$.

    Each rotation $g^k$ is a permutation of order $\frac{n}{gcd(n,k)}$ and thus gives us $gcd(k,n)$ orbits. And this and Burnside's lemma give us:

    $|A^{12} / G|$ = $\frac{1}{12} \sum\limits_{i=1}^{12} n^{gcd(i,12)} = \frac{n^1+n^2+n^3+n^4+n^1+n^6+n^1+n^4+n^3+n^2+n^1}{12} = \frac{4n^1+2n^2+2n^3+2n^4+n^6}{12}$.



\end{document}
